\newpage
\section{Crypto - Security analysis}
Let us assume that our system is deployed with multiple sensor nodes that all
share the same key.
We ask you to analyze the \textbf{security} properties of the system against the
following attacks:
\begin{itemize}
    \item Forgery of packets using a wrong key.
    \item CBC-MAC length extension attack.
    \item CBC-MAC padding attack.
    \item Transmitter modification attack (capture a packet sent by the device
        while jamming the receiver, and re-emit the packet, while changing the
        \texttt{emitter\_id}).
    \item Packet replay attack (capture a packet sent by a device, and re-emit
        it later).
    \item Packet delay attack (capture a packet sent by the device while
        jamming the receiver, and re-emit the packet later).
    \item Eavesdropping attack: capture the packets and extract information.
    \item Timing attack against the AES implementation.\footnote{%
            For more information of timing attacks, see~\url{%
                https://en.wikipedia.org/wiki/Timing_attack
            } and, for more details,
            \url{http://citeseerx.ist.psu.edu/viewdoc/download?doi=10.1.1.42.679&rep=rep1&type=pdf}
            (the AES block cipher was originally named Rijndael).
        }
\end{itemize}

\noindent For each of those attacks, describe
\begin{itemize}
    \item the plausible goals of an adversary that would mount such an attack
        (perform this analysis while considering that the system monitors a
        forest),
    \item whether the attack is possible to mount, and what capabilities would the
        adversary need to mount the attack (Justify your claim based on your
        understanding of the system and support your claims by analysis of the
        code (running on the MCU or on the receiver) or by experimental
        evidence.),
    \item if the attack is possible to mount, analyze whether protecting
        against it would require a packet format change (with respect to the format
        given in H6,
        or if the packet can be kept unchanged, and only the emitter or
        receiver implementation have to be changed.
\end{itemize}
\noindent The report concerns security at the cryptographic level. Thus, assume for all cases the following :
\begin{itemize}
\item The adversary knows every detail of the implementation, packet format and so on. Only the secret shared key is hidden from the adversary.
\item The receiver node is able to process all incoming packets, therefore jamming and denial of service are out of scope.
\end{itemize}
Finally, be brief and limit yourselves to the one attack listed above at a time (i.e. don't use them as basis for other attacks.). Every attack can be analyzed and justified in a couple of lines.

\begin{bclogo}[couleur = gray!20, arrondi = 0.2, logo=\bcinfo]{AES timing attack vulnerability}
    If the execution time of the implementation of a cryptographic primitive
    depends on the key, then it is vulnerable to timing attacks.
    In order to test this property for the AES implementation you use, you need
    perfectly repeatable (i.e., noise-free) measurements of the execution time
    of the AES encryption function. (The \texttt{\_\_disable\_irq();} and
    \texttt{\_\_enable\_irq();} macros may be useful.)
    Then measure the execution time for various keys and quantify its variability.
\end{bclogo}
