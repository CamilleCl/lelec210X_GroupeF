\section{Demonstration}

%\subsection{Demonstration}

On next Wednesday's session, we expect from each group a quick demonstration showing a correct reception of 5 packets transmitted over the air using the radio evaluation mode
at the default Tx power level of -16~dBm, with both the hardware LPF and PPD enabled.

\begin{comment}
\subsection{Report}

Please upload a report of \textbf{maximum} 2 pages with
\begin{enumerate}
    \item The output of the python testbench of the preamble detector (\texttt{3\_compare.py}), as well as your implementation of the Absolute-value norm.

    \item A summary of the resource usage (logic elements) and worst slack (setup) for the different compilation performed :
    \begin{enumerate}
        \item Initial design without register retiming and embedded multipliers.
        \item Initial design with register retiming and embedded multipliers.
        \item Final design with Absolute-value norm estimator.
    \end{enumerate}

    \item A breakdown of the resource usage for the final design including the Absolute-value norm. Focus mainly on logic cells/elements, for both the overall design (\texttt{lms7\_trx\_top}) and for the LimeSuite digital signal processing (\texttt{lms\_dsp)}. You can present it as a pie chart or an histogram. In both case, try to highlight the FIR (low-pass filter) and the preamble detector. You can find the required informations either in the \textit{Hierarchy} tab of the \textit{Project Navigator} view, or in the \textit{Analysis and Synthesis/Resource Utilization by Entity} tab of the compilation report.
\end{enumerate}
\end{comment}
