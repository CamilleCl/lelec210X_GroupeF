\documentclass[a4paper,11pt]{article}
\usepackage[utf8]{inputenc}
\usepackage[T1]{fontenc}
\usepackage{amsmath}
\usepackage{mathtools}
\usepackage{amsfonts}
\usepackage{amssymb}
\usepackage{graphicx}
\usepackage{multicol}
\usepackage{array}
\usepackage{float}
\usepackage{epstopdf}
\usepackage{caption}
\usepackage{subcaption}
\usepackage{gensymb}
\usepackage[bottom]{footmisc}
\usepackage{appendix}
\usepackage{pdfpages}
\usepackage{todonotes}
\usepackage{mathpazo}
\usepackage{titleps}
\usepackage{color}
\usepackage{xcolor}
\usepackage{colortbl}
%\usepackage{siunitx}
\usepackage{pdflscape}
\usepackage{cancel}

\usepackage[skins]{tcolorbox}
\usepackage{sectsty}
\usepackage[arrowmos]{circuitikz}
\usepackage{pgfplots}
\usepackage{blindtext}
\usepackage[inner=2cm,outer=2cm,top=2.5cm,bottom=2.5cm]{geometry}
\usepackage{todonotes}
\usepackage{hyperref}
\usepackage{url}
\usepackage{adjustbox}
\usepackage{tabularx}
\usepackage{booktabs}
\usepackage{multirow}
%\usepackage[table,xcdraw]{xcolor}

\graphicspath{{figures/}}
\sectionfont{\large}
\subsectionfont{\normalsize}



%%%%%%%%%%%%%%%%%%%
% HANDS-ON NUMBER
\newcommand\handsOnN{0b}
% WEEK NUMBER
\newcommand\weekN{0}
%%%%%%%%%%%%%%%%%%%

\newpagestyle{main}{
	\sethead[LELEC2102: Manipulating arrays in C][][Week \weekN]{LELEC2102: Manipulating arrays in C}{}{Week \weekN}
	\headrule
    \setfoot[][\thepage][]{}{\thepage}{}
}


\newcommand{\horrule}[1]{\rule{\linewidth}{#1}} % Create horizontal rule command with 1 argument of height

\begin{document}
\renewcommand{\figurename}{Fig.}

\renewcommand{\thepage}{\arabic{page}}
\setcounter{page}{1}
\pagestyle{main}
\newpage \clearpage

\begin{center}
\begin{huge}
Preliminary work : Manipulating arrays in C\\
\end{huge}
\vspace{0.3cm}
%\textit{TA 1, TA 2}
\end{center}
\section{Introduction}
During the semester, you will use C extensively, and after this homework you should be ready to cope with the code of the project.\\

In this homework, you will perform a base 64 encoding and decoding of an array of bytes. You will probably need a refreshment of your C knowledge, in which case please read the references cited below.\\
It is necessary that each member of your group is up to date with the concepts. Therefore, we ask you to do this homework independently and not in groups.\footnote{Of course it is not forbidden to get help from your group or classmates, but in the end, each one of you should have its own working and understood implementation.}\\
\section{The base 64 encoding}
The base 64 encoding is a 6-bit encoding that uses capital letters, small letters, numbers and a few basic symbols in order to obtain 64 different characters as shown in \autoref{tab:b64}. If the content that you encode is not a multiple of 24 bits (or 3 bytes or 4 base 64 symbols), then you must use the equals (=) symbol as padding to reach a multiple of 24 bits.

\begin{table}[h]
	\begin{tabular}{ccc|ccc|ccc|ccc}
		\rowcolor[HTML]{EAECF0}
		{\color[HTML]{202122} \textbf{Index}} & {\color[HTML]{202122} \textbf{Binary}} & {\color[HTML]{202122} \textbf{Char}} & {\color[HTML]{202122} \textbf{Index}} & {\color[HTML]{202122} \textbf{Binary}} & {\color[HTML]{202122} \textbf{Char}} & {\color[HTML]{202122} \textbf{Index}} & {\color[HTML]{202122} \textbf{Binary}} & {\color[HTML]{202122} \textbf{Char}} & {\color[HTML]{202122} \textbf{Index}} & {\color[HTML]{202122} \textbf{Binary}} & {\color[HTML]{202122} \textbf{Char}} \\
		\rowcolor[HTML]{F8F9FA}
		{\color[HTML]{202122} 0}              & {\color[HTML]{202122} 000000}          & {\color[HTML]{202122} A}             & {\color[HTML]{202122} 16}             & {\color[HTML]{202122} 010000}          & {\color[HTML]{202122} Q}             & {\color[HTML]{202122} 32}             & {\color[HTML]{202122} 100000}          & {\color[HTML]{202122} g}             & {\color[HTML]{202122} 48}             & {\color[HTML]{202122} 110000}          & {\color[HTML]{202122} w}             \\
		\rowcolor[HTML]{F8F9FA}
		{\color[HTML]{202122} 1}              & {\color[HTML]{202122} 000001}          & {\color[HTML]{202122} B}             & {\color[HTML]{202122} 17}             & {\color[HTML]{202122} 010001}          & {\color[HTML]{202122} R}             & {\color[HTML]{202122} 33}             & {\color[HTML]{202122} 100001}          & {\color[HTML]{202122} h}             & {\color[HTML]{202122} 49}             & {\color[HTML]{202122} 110001}          & {\color[HTML]{202122} x}             \\
		\rowcolor[HTML]{F8F9FA}
		{\color[HTML]{202122} 2}              & {\color[HTML]{202122} 000010}          & {\color[HTML]{202122} C}             & {\color[HTML]{202122} 18}             & {\color[HTML]{202122} 010010}          & {\color[HTML]{202122} S}             & {\color[HTML]{202122} 34}             & {\color[HTML]{202122} 100010}          & {\color[HTML]{202122} i}             & {\color[HTML]{202122} 50}             & {\color[HTML]{202122} 110010}          & {\color[HTML]{202122} y}             \\
		\rowcolor[HTML]{F8F9FA}
		{\color[HTML]{202122} 3}              & {\color[HTML]{202122} 000011}          & {\color[HTML]{202122} D}             & {\color[HTML]{202122} 19}             & {\color[HTML]{202122} 010011}          & {\color[HTML]{202122} T}             & {\color[HTML]{202122} 35}             & {\color[HTML]{202122} 100011}          & {\color[HTML]{202122} j}             & {\color[HTML]{202122} 51}             & {\color[HTML]{202122} 110011}          & {\color[HTML]{202122} z}             \\
		\rowcolor[HTML]{F8F9FA}
		{\color[HTML]{202122} 4}              & {\color[HTML]{202122} 000100}          & {\color[HTML]{202122} E}             & {\color[HTML]{202122} 20}             & {\color[HTML]{202122} 010100}          & {\color[HTML]{202122} U}             & {\color[HTML]{202122} 36}             & {\color[HTML]{202122} 100100}          & {\color[HTML]{202122} k}             & {\color[HTML]{202122} 52}             & {\color[HTML]{202122} 110100}          & {\color[HTML]{202122} 0}             \\
		\rowcolor[HTML]{F8F9FA}
		{\color[HTML]{202122} 5}              & {\color[HTML]{202122} 000101}          & {\color[HTML]{202122} F}             & {\color[HTML]{202122} 21}             & {\color[HTML]{202122} 010101}          & {\color[HTML]{202122} V}             & {\color[HTML]{202122} 37}             & {\color[HTML]{202122} 100101}          & {\color[HTML]{202122} l}             & {\color[HTML]{202122} 53}             & {\color[HTML]{202122} 110101}          & {\color[HTML]{202122} 1}             \\
		\rowcolor[HTML]{F8F9FA}
		{\color[HTML]{202122} 6}              & {\color[HTML]{202122} 000110}          & {\color[HTML]{202122} G}             & {\color[HTML]{202122} 22}             & {\color[HTML]{202122} 010110}          & {\color[HTML]{202122} W}             & {\color[HTML]{202122} 38}             & {\color[HTML]{202122} 100110}          & {\color[HTML]{202122} m}             & {\color[HTML]{202122} 54}             & {\color[HTML]{202122} 110110}          & {\color[HTML]{202122} 2}             \\
		\rowcolor[HTML]{F8F9FA}
		{\color[HTML]{202122} 7}              & {\color[HTML]{202122} 000111}          & {\color[HTML]{202122} H}             & {\color[HTML]{202122} 23}             & {\color[HTML]{202122} 010111}          & {\color[HTML]{202122} X}             & {\color[HTML]{202122} 39}             & {\color[HTML]{202122} 100111}          & {\color[HTML]{202122} n}             & {\color[HTML]{202122} 55}             & {\color[HTML]{202122} 110111}          & {\color[HTML]{202122} 3}             \\
		\rowcolor[HTML]{F8F9FA}
		{\color[HTML]{202122} 8}              & {\color[HTML]{202122} 001000}          & {\color[HTML]{202122} I}             & {\color[HTML]{202122} 24}             & {\color[HTML]{202122} 011000}          & {\color[HTML]{202122} Y}             & {\color[HTML]{202122} 40}             & {\color[HTML]{202122} 101000}          & {\color[HTML]{202122} o}             & {\color[HTML]{202122} 56}             & {\color[HTML]{202122} 111000}          & {\color[HTML]{202122} 4}             \\
		\rowcolor[HTML]{F8F9FA}
		{\color[HTML]{202122} 9}              & {\color[HTML]{202122} 001001}          & {\color[HTML]{202122} J}             & {\color[HTML]{202122} 25}             & {\color[HTML]{202122} 011001}          & {\color[HTML]{202122} Z}             & {\color[HTML]{202122} 41}             & {\color[HTML]{202122} 101001}          & {\color[HTML]{202122} p}             & {\color[HTML]{202122} 57}             & {\color[HTML]{202122} 111001}          & {\color[HTML]{202122} 5}             \\
		\rowcolor[HTML]{F8F9FA}
		{\color[HTML]{202122} 10}             & {\color[HTML]{202122} 001010}          & {\color[HTML]{202122} K}             & {\color[HTML]{202122} 26}             & {\color[HTML]{202122} 011010}          & {\color[HTML]{202122} a}             & {\color[HTML]{202122} 42}             & {\color[HTML]{202122} 101010}          & {\color[HTML]{202122} q}             & {\color[HTML]{202122} 58}             & {\color[HTML]{202122} 111010}          & {\color[HTML]{202122} 6}             \\
		\rowcolor[HTML]{F8F9FA}
		{\color[HTML]{202122} 11}             & {\color[HTML]{202122} 001011}          & {\color[HTML]{202122} L}             & {\color[HTML]{202122} 27}             & {\color[HTML]{202122} 011011}          & {\color[HTML]{202122} b}             & {\color[HTML]{202122} 43}             & {\color[HTML]{202122} 101011}          & {\color[HTML]{202122} r}             & {\color[HTML]{202122} 59}             & {\color[HTML]{202122} 111011}          & {\color[HTML]{202122} 7}             \\
		\rowcolor[HTML]{F8F9FA}
		{\color[HTML]{202122} 12}             & {\color[HTML]{202122} 001100}          & {\color[HTML]{202122} M}             & {\color[HTML]{202122} 28}             & {\color[HTML]{202122} 011100}          & {\color[HTML]{202122} c}             & {\color[HTML]{202122} 44}             & {\color[HTML]{202122} 101100}          & {\color[HTML]{202122} s}             & {\color[HTML]{202122} 60}             & {\color[HTML]{202122} 111100}          & {\color[HTML]{202122} 8}             \\
		\rowcolor[HTML]{F8F9FA}
		{\color[HTML]{202122} 13}             & {\color[HTML]{202122} 001101}          & {\color[HTML]{202122} N}             & {\color[HTML]{202122} 29}             & {\color[HTML]{202122} 011101}          & {\color[HTML]{202122} d}             & {\color[HTML]{202122} 45}             & {\color[HTML]{202122} 101101}          & {\color[HTML]{202122} t}             & {\color[HTML]{202122} 61}             & {\color[HTML]{202122} 111101}          & {\color[HTML]{202122} 9}             \\
		\rowcolor[HTML]{F8F9FA}
		{\color[HTML]{202122} 14}             & {\color[HTML]{202122} 001110}          & {\color[HTML]{202122} O}             & {\color[HTML]{202122} 30}             & {\color[HTML]{202122} 011110}          & {\color[HTML]{202122} e}             & {\color[HTML]{202122} 46}             & {\color[HTML]{202122} 101110}          & {\color[HTML]{202122} u}             & {\color[HTML]{202122} 62}             & {\color[HTML]{202122} 111110}          & {\color[HTML]{202122} +}             \\
		\rowcolor[HTML]{F8F9FA}
		{\color[HTML]{202122} 15}             & {\color[HTML]{202122} 001111}          & {\color[HTML]{202122} P}             & {\color[HTML]{202122} 31}             & {\color[HTML]{202122} 011111}          & {\color[HTML]{202122} f}             & {\color[HTML]{202122} 47}             & {\color[HTML]{202122} 101111}          & {\color[HTML]{202122} v}             & {\color[HTML]{202122} 63}             & {\color[HTML]{202122} 111111}          & {\color[HTML]{202122} /}             \\
		\rowcolor[HTML]{F8F9FA}
		\multicolumn{2}{c}{\cellcolor[HTML]{ECECEC}{\color[HTML]{2C2C2C} Padding}}     & {\color[HTML]{202122} =}             & {\color[HTML]{202122} }               & {\color[HTML]{202122} }                & {\color[HTML]{202122} }              & {\color[HTML]{202122} }               & {\color[HTML]{202122} }                & {\color[HTML]{202122} }              & {\color[HTML]{202122} }               & {\color[HTML]{202122} }                & {\color[HTML]{202122} }
	\end{tabular}
	\caption{Base 64 encoding table}
	\label{tab:b64}
\end{table}

Concretely, we ask you to implement the functions \texttt{b64\_encode} and \texttt{b64\_decode} in C and test them using the given reference python code.

In order to compile C code, the easiest way is to run GCC on your virtual machine with the following command \hspace{3mm}\texttt{gcc source\_file.c -o output\_executable\_name}\hspace{3mm}, then run your program with \hspace{3mm} \texttt{./output\_executable\_name}\hspace{3mm}.\\

You can pass command-line arguments to each program and print the result in the terminal.

\newpage
\section{References}
There are many more resources to learn C, we don't obligate you to use any of them but we expect you to be familiar with the language in particular with array manipulation, pointers, structures and bit manipulations. We propose here a selection of references that cover the points listed previously. \\

In French (available on Moodle):
\begin{itemize}
	\item LEPL1503 : Introduction au langage C : Sections 2.2.3 to 2.2.6
	\item Programmation en langage C, Anne Canteaut :  Section 1.7
\end{itemize}

In English :
\begin{itemize}
	\item Smaller C - Lean Code for Small Machines, Marc Loy : Chapters 4 and 6\footnote{A 10 day free trial is available for this book in its online format at : \url{https://learning.oreilly.com/library/view/smaller-c/9781098100322/}}
\end{itemize}


\end{document}
