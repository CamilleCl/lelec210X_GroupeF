\section{Report R6}

The report R6 (due on \textbf{Nov 22, Monday 6.30 PM}). We expect you to:
\begin{itemize}
    \item Analyze with a graph, the number of cycles taken by the MAC function
        per authenticated byte for different message lengths : 1 byte up to 200
        bytes with steps of 1 byte. Do this for the different compiler
        optimization levels:
        \texttt{-O0},
        \texttt{-O1},
        \texttt{-O2},
        \texttt{-O3},
        \texttt{-Os}.\footnote{%
            To understand what these optimization levels do (as well as
            \texttt{-Og}), see
            \url{https://gcc.gnu.org/onlinedocs/gcc/Optimize-Options.html}.
            The optimizations controlled by these flags happen at the compiler
            level, and each file is compiled individually.

            Since optimization is based on gathering information on the
            code and how it is used, this limited (file-level) view may lead to
            missed optimization opportunities.
            Link-time optimization (LTO) is a technique that enables
            project-level view for optimizations (see
            \url{https://gcc.gnu.org/onlinedocs/gcc/Optimize-Options.html})
            and that you might want to use at some stage of your project.
            To add such custom compiler and linker flags, use the
            \textbf{Miscellaneous} section of the Compiler/Linker settings.
        }
    \item For the same optimization levels, analyze the code size of the
        authentication part of the project by comparing the total code size with
        and without authentication.
    \item Briefly discuss your results.
\end{itemize}
