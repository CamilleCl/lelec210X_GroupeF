\section*{Material}

\begin{comment}[couleur = gray!20, arrondi = 0.2, logo=\bcinfo]{}
\vspace{0.2cm}
\end{comment}
For this hands-on session, you will need:
\begin{itemize}
    \item A computer with Python installed and running;
    \item The simulation files (in Python) provided by the teaching team, available on Moodle.
\end{itemize}
%\vspace{0.01cm}
\begin{bclogo}[couleur = gray!20, arrondi = 0.2, logo=\bcinfo]{Deliverables for this hands-on}
\bccrayon: This icon indicates parts of the hands-on where you have to actively complete some codes or run a simulation.\\

This hands-on is particularly detailed to help you understand the provided simulation framework. This does not mean that you have a lot of work to do! Indeed, we only expect you to complete two functions in the simulation framework and perform some simulations.

After the two hands-on sessions, each group will have to write a small report containing the completed functions as some simulation results. All the guidelines for the report are given in the last section of this document.

\end{bclogo}
