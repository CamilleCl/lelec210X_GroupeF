\section*{Introduction}
%
In this hands-on session, you will pass the computation of your feature vectors to the MCU (transmitter side).
The operations applied on the data are essentially the same as what you did during the first hands-on H1.
However there are two essential differences in this hands-on you will rapidly observe:
\begin{itemize}
    \item \textbf{The programming language}: you abandon the very easy and comfortable but high-level and slow \emph{Python} for low-level and fast \emph{C-code}.
    \item \textbf{Data representation}: for speed and memory issues due to the limitations of your MCU, you jump from the 32-bit \emph{floating point} representation to the 16-bit \emph{fixed point} representation.
\end{itemize}
%
This language switching implies new packages to handle for the use of fast implementation routines.
To start this hands-on session:
\begin{itemize}
    \item Open the \textbf{mcu/hands\_on\_feature\_vectors} project from File System on your \textbf{STM32CubeIDE}.
    \item On STM32CubeIDE, open \textbf{hands\_on\_feature\_vectors.ioc} and click on \emph{Project$\rightarrow$Generate code}.
    \item In \emph{Code/}, open \textbf{config.h} and \textbf{spectrogram\_tables.h} from \emph{Inc/}, and \textbf{main.c}, \textbf{spectrogram.c} from \emph{Src/}.
\end{itemize}
